\documentclass[10pt]{beamer}

\usetheme[progressbar=frametitle]{metropolis}
\usepackage{appendixnumberbeamer}

\usepackage{booktabs}
\usepackage{ulem}

\usepackage[scale=2]{ccicons}
\usepackage{pgfplots}
\usepgfplotslibrary{dateplot}
\usepackage{xspace}
\newcommand{\themename}{\textbf{\textsc{metropolis}}\xspace}
\usepackage[
backend=biber,
style=authoryear,
sorting=ynt
]{biblatex}
\addbibresource{biblio.bib}
\usepackage[justification=centering]{caption}
\captionsetup{labelformat=empty,labelsep=none}
\usepackage{enumerate}   
\usepackage[T1]{fontenc}
\usepackage{gensymb}
\setbeamertemplate{section in toc}[sections numbered]
\definecolor{cornflowerblue}{rgb}{0.39, 0.58, 0.93}

\hypersetup{colorlinks,linkcolor=,urlcolor=cornflowerblue}

\makeatletter
\setlength{\metropolis@titleseparator@linewidth}{2pt}
\setlength{\metropolis@progressonsectionpage@linewidth}{2pt}
\setlength{\metropolis@progressinheadfoot@linewidth}{2pt}
\makeatother

\title{Git reminder}
\date{11/05/2023}
\author{Nicolas Barrier \and Witold Podlejski \and Criscely
Lujan-Paredes}
\date{}

\begin{document}
\frame{\titlepage}
\ifdefined\Shaded\renewenvironment{Shaded}{\begin{tcolorbox}[sharp corners, frame hidden, boxrule=0pt, breakable, borderline west={3pt}{0pt}{shadecolor}, enhanced, interior hidden]}{\end{tcolorbox}}\fi

\begin{frame}{Basic commands}
    \begin{itemize}
        \item \textbf{git init}: initialise a git project (create .git folder)
        \item \textbf{git add [files]}: add files to list of tracked files
        \vspace{0.2cm}
        \item \textbf{git commit -m "message"}: validate locally a version of the project
        \vspace{0.2cm}
        \item \textbf{git status}: see the unvalidated and  untracked changes
        \vspace{0.2cm}
        \item \textbf{git checkout [commit]}: load the project version corresponding to the index
        \vspace{0.2cm}
        \item \textbf{git pull}: import the changes from remote project to local
        \vspace{0.2cm}
        \item \textbf{git push}: export the changes from local project to the remote server
        \vspace{0.2cm}
    \end{itemize}
\end{frame}

\begin{frame}[fragile]{Git configuration (Mandatory)}



\begin{itemize}
\item Configure your \textbf{identity}: \texttt{git\ config\ -\/-global\ user.name\ "Firstname\ Lastname"}
\vspace{0.5cm}
\item Configure your \textbf{e-mail}: \texttt{git\ config\ -\/-global\ user.email\ "myadresse@ird.fr"}
\end{itemize}

\end{frame}


\begin{frame}{Branch handling}
    \begin{itemize}
        \item \textbf{git branch [branch\_name]}: create a new branch (but you remain on the previous branch)
        \item \textbf{git branch -b [branch\_name]}: create a new branch and move to this newly created branch
        \vspace{0.3cm}
        \item \textbf{git checkout [branch\_name]}: move to the corresponding branch
        \vspace{0.3cm}
        \item \textbf{git merge [branch\_name1] [branch\_name2]}: merge two different branch, you may need to resolve version conflict.
        \item \textbf{git branch -d [branch\_name1]}: delete a branch (safe mode)
        \item \textbf{git branch -D [branch\_name1]}: delete a branch (unsafe mode)
    
        
    \end{itemize}
\end{frame}


\begin{frame}{Linking local with remote}
    \begin{itemize}
        \item \textbf{git clone [URL]}: Import an existing project from remote server.
        \vspace{0.5cm}
        \item \textbf{git remote add origin [URL]}: link directly the local repository with a remote
    \end{itemize}
\end{frame}

\begin{frame}{Authentication of your computer and the remote server}
    \begin{itemize}
        \item \textbf{SSH key}: easy way on Linux distributions
        \begin{itemize}
            % \item Type \texttt{ssh-keygen -t ed25519} on your computer
            % \item Create a new ssh key on the remote server (GitHub)
            % \item Copy the content of ~/.ssh/id\_ed25519\textbf{.pub} on the remote client
            \item Tuto here: \url{https://jdblischak.github.io/2014-09-18-chicago/novice/git/05-sshkeys.html}
        \end{itemize}
        \vspace{0.2cm}
        \item \textbf{Authentication Token}
            \begin{itemize}
            \item Tuto here:  \url{https://docs.github.com/en/authentication/keeping-your-account-and-data-secure/creating-a-personal-access-token}
        \end{itemize}
    \end{itemize}
\end{frame}

\begin{frame}{Good practices}
    \begin{itemize}
        \item \textbf{Pull} before any work on the project
        \vspace{0.2cm}
        \item \textbf{Commit} as frequently as possible
        \vspace{0.2cm}
        \item Write explicit \textbf{commit message}
        \vspace{0.2cm}
        \item \textbf{Push} regularly
        
    \end{itemize}
\end{frame}

\begin{frame}{IDE (graphical user interface) with Git}
    \begin{itemize}
        \item \textbf{R}
        \begin{itemize}
            \item RStudio
            \item Visual Studio Code
        \end{itemize}
        \vspace{0.2cm}
        \item \textbf{Python}
        \begin{itemize}
            \item Spyder
            \item Visual Studio Code
            \item Pycharm (all JetBrain softwares)
        \end{itemize}
    \end{itemize}
\end{frame}


\end{document}
